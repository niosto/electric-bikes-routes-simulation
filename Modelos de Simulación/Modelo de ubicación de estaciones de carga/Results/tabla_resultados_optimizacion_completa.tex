% ============================================================================
% TABLA 1: RESULTADOS DE OPTIMIZACIÓN DE ESTACIONES DE CARGA
% ============================================================================
\begin{table}[htbp]
\centering
\caption{Resultados de Optimización de Ubicaciones de Estaciones de Carga para Vehículos Eléctricos}
\label{tab:optimizacion_estaciones}
\begin{tabular}{lcccc}
\toprule
\textbf{Ciudad} & \textbf{Total Estaciones} & \textbf{Standard Charger} & \textbf{High Capacity Charger} & \textbf{Diversidad Tecnológica} \\
\midrule
Bogotá & 9 & 9 & 0 & No \\
Medellín & 10 & 9 & 1 & Sí \\
Valle de Aburrá & 9 & 9 & 0 & No \\
\midrule
\textbf{Total} & \textbf{28} & \textbf{27} & \textbf{1} & \\
\bottomrule
\end{tabular}
\begin{flushleft}
\footnotesize
\textit{Nota:} Los resultados se obtuvieron mediante optimización MILP (Mixed Integer Linear Programming) 
basada en clustering DBSCAN jerárquico de orígenes y destinos de viajes. La diversidad tecnológica 
indica si se utilizaron ambos tipos de cargadores en la solución óptima. El método considera 
restricciones de presupuesto, capacidad, cobertura y asignación de demanda.
\end{flushleft}
\end{table}

% ============================================================================
% TABLA 2: DISTRIBUCIÓN GEOGRÁFICA DE ESTACIONES
% ============================================================================
\begin{table}[htbp]
\centering
\caption{Distribución Geográfica de Estaciones de Carga Óptimas}
\label{tab:distribucion_geografica}
\begin{tabular}{lcc}
\toprule
\textbf{Ciudad} & \textbf{Rango Latitud (°)} & \textbf{Rango Longitud (°)} \\
\midrule
Bogotá & 0.054 & 0.027 \\
Medellín & 0.081 & 0.064 \\
Valle de Aburrá & 0.099 & 0.064 \\
\bottomrule
\end{tabular}
\begin{flushleft}
\footnotesize
\textit{Nota:} Los rangos geográficos indican la extensión espacial de las estaciones 
óptimas en cada ciudad, calculados como la diferencia entre las coordenadas máximas 
y mínimas de las estaciones seleccionadas.
\end{flushleft}
\end{table}

% ============================================================================
% TABLA 3: RESUMEN DEL PROCESO DE CLUSTERING Y OPTIMIZACIÓN
% ============================================================================
\begin{table}[htbp]
\centering
\caption{Resumen del Proceso de Clustering DBSCAN y Optimización MILP}
\label{tab:proceso_clustering}
\begin{tabular}{lcc}
\toprule
\textbf{Etapa del Proceso} & \textbf{Método} & \textbf{Propósito} \\
\midrule
1. Limpieza de Datos & Filtrado geográfico & Eliminar viajes fuera de zona \\
2. Zonificación & Filtrado por límites administrativos & Validar viajes dentro de la ciudad \\
3. Clustering de Orígenes & DBSCAN Jerárquico (2 etapas) & Identificar zonas de alta demanda \\
3. Clustering de Destinos & DBSCAN Jerárquico (2 etapas) & Identificar zonas de destino frecuente \\
4. Agregación de Flujos & Agrupación O-D & Consolidar patrones de viaje \\
5. Muestreo Estratificado & Muestreo proporcional & Seleccionar rutas representativas \\
6. Simulación Física & OpenRouteService API & Calcular consumo energético real \\
7. Cálculo de Costos de Acceso & Pre-computación & Determinar viabilidad de estaciones \\
8. Optimización MILP & Solver MOSEK/CBC & Encontrar ubicaciones óptimas \\
\bottomrule
\end{tabular}
\begin{flushleft}
\footnotesize
\textit{Nota:} El proceso completo integra técnicas de machine learning (clustering) 
con optimización matemática para determinar las ubicaciones óptimas de estaciones 
de carga considerando demanda real, restricciones físicas y económicas.
\end{flushleft}
\end{table}


