\documentclass[12pt]{article}
\usepackage[utf8]{inputenc}
\usepackage[spanish]{babel}
\usepackage{geometry}
\usepackage{longtable}
\usepackage{booktabs}
\usepackage{array}
\usepackage{xcolor}
\usepackage{colortbl}

\geometry{a4paper, margin=2cm}

\title{Coordenadas de Estaciones de Carga\\(Casos de Estudio: Bogotá, Medellín y Valle de Aburrá)}
\author{Sistema de Optimización de Infraestructura de Carga}
\date{\today}

\begin{document}

\maketitle

\section{Resumen}
Este documento presenta las coordenadas geográficas (latitud y longitud) de todas las estaciones de carga optimizadas para los tres casos de estudio: Bogotá, Medellín y Valle de Aburrá.

\begin{table}[h]
\centering
\caption{Resumen de Estaciones por Caso de Estudio}
\begin{tabular}{lcc}
\toprule
\textbf{Caso de Estudio} & \textbf{Número de Estaciones} & \textbf{Tecnologías} \\
\midrule
Bogotá & 8 & Cargador Alta Capacidad: 2, Cargador Estándar: 5, Intercambio de Baterías: 1 \\
Medellín & 8 & Cargador Alta Capacidad: 1, Cargador Estándar: 7 \\
Valle de Aburrá & 10 & Cargador Estándar: 10 \\
\bottomrule
\end{tabular}
\end{table}

\section{Coordenadas Detalladas de Estaciones}

\begin{longtable}{p{2.5cm}p{3.5cm}p{4cm}cc}
\caption{Coordenadas de todas las estaciones de carga optimizadas}\\
\toprule
\textbf{Caso de Estudio} & \textbf{ID Estación} & \textbf{Tecnología} & \textbf{Latitud (°)} & \textbf{Longitud (°)} \\
\midrule
\endfirsthead

\multicolumn{5}{c}{{\tablename\ \thetable{} -- continuación de la página anterior}} \\
\toprule
\textbf{Caso de Estudio} & \textbf{ID Estación} & \textbf{Tecnología} & \textbf{Latitud (°)} & \textbf{Longitud (°)} \\
\midrule
\endhead

\midrule
\multicolumn{5}{r}{{Continúa en la página siguiente}} \\
\endfoot

\bottomrule
\endlastfoot

% BOGOTÁ
Bogotá & ST-BOG-095-037 & Cargador Alta Capacidad & 4.590993 & -74.109941 \\
Bogotá & ST-BOG-095-039 & Cargador Alta Capacidad & 4.590993 & -74.091759 \\
Bogotá & ST-BOG-095-041 & Cargador Estándar & 4.590993 & -74.073578 \\
Bogotá & ST-BOG-097-037 & Intercambio de Baterías & 4.609011 & -74.109941 \\
Bogotá & ST-BOG-097-040 & Cargador Estándar & 4.609011 & -74.082669 \\
Bogotá & ST-BOG-100-040 & Cargador Estándar & 4.636039 & -74.082669 \\
Bogotá & ST-BOG-101-038 & Cargador Estándar & 4.645048 & -74.100850 \\
Bogotá & ST-BOG-102-040 & Cargador Estándar & 4.654057 & -74.082669 \\

% MEDELLÍN
Medellín & ST-MED-001-016 & Cargador Estándar & 6.177337 & -75.569288 \\
Medellín & ST-MED-003-017 & Cargador Estándar & 6.195355 & -75.560197 \\
Medellín & ST-MED-007-016 & Cargador Estándar & 6.231391 & -75.569288 \\
Medellín & ST-MED-007-020 & Cargador Estándar & 6.231391 & -75.532924 \\
Medellín & ST-MED-009-013 & Cargador Estándar & 6.249409 & -75.596561 \\
Medellín & ST-MED-009-017 & Cargador Estándar & 6.249409 & -75.560197 \\
Medellín & ST-MED-009-019 & Cargador Estándar & 6.249409 & -75.542015 \\
Medellín & ST-MED-010-015 & Cargador Alta Capacidad & 6.258418 & -75.578379 \\

% VALLE DE ABURRÁ
Valle de Aburrá & ST-VAL-027-014 & Cargador Estándar & 6.226001 & -75.587470 \\
Valle de Aburrá & ST-VAL-028-016 & Cargador Estándar & 6.235010 & -75.569288 \\
Valle de Aburrá & ST-VAL-028-020 & Cargador Estándar & 6.235010 & -75.532924 \\
Valle de Aburrá & ST-VAL-029-014 & Cargador Estándar & 6.244019 & -75.587470 \\
Valle de Aburrá & ST-VAL-030-016 & Cargador Estándar & 6.253028 & -75.569288 \\
Valle de Aburrá & ST-VAL-030-019 & Cargador Estándar & 6.253028 & -75.542015 \\
Valle de Aburrá & ST-VAL-032-017 & Cargador Estándar & 6.271046 & -75.560197 \\
Valle de Aburrá & ST-VAL-033-013 & Cargador Estándar & 6.280055 & -75.596561 \\
Valle de Aburrá & ST-VAL-034-019 & Cargador Estándar & 6.289064 & -75.542015 \\
Valle de Aburrá & ST-VAL-035-016 & Cargador Estándar & 6.298073 & -75.569288 \\

\end{longtable}

\section{Notas}
\begin{itemize}
    \item Las coordenadas están en el sistema de referencia WGS84 (EPSG:4326).
    \item La latitud es positiva hacia el norte (valores positivos indican hemisferio norte).
    \item La longitud es negativa hacia el oeste (valores negativos indican hemisferio oeste para Colombia).
    \item Las tecnologías disponibles son:
    \begin{itemize}
        \item \textbf{Cargador Estándar}: Cargador convencional con capacidad moderada.
        \item \textbf{Cargador Alta Capacidad}: Cargador rápido con mayor capacidad de servicio.
        \item \textbf{Intercambio de Baterías}: Estación que permite el intercambio rápido de baterías.
    \end{itemize}
    \item Total de estaciones: 26 estaciones distribuidas en los tres casos de estudio.
\end{itemize}

\end{document}

